\documentclass{article}
\usepackage{graphicx}
\usepackage{amssymb,amsmath}

\begin{document}

\title{Introduction to \LaTeX{}}
\author{Author's Name}

\maketitle

\begin{abstract}
The abstract text goes here.
\end{abstract}

\section{Introduction}


\begin{equation}
\label{eq:1}
\dot{x} = rx\left(1 - \frac{x}{K}\right)
\end{equation}

\begin{equation}
\label{eq:2}
x(t) = \frac{KPe^{rt}}{K + P(e^{rt}-1)},
\end{equation}

\begin{equation}
\dot{C} = rC\left(1 - \frac{C}{K}\right)
\end{equation}

\begin{equation}
C(t) = \frac{KC_{0}e^{rt}}{K + C_{0}(e^{rt}-1)},
\end{equation}

\begin{subequations}
	\begin{align}
	&C{i} - Cells\\
	&N{i} - Nutrients
	\end{align}
\end{subequations}

\begin{subequations}
	\label{eq:9}
	\begin{align}
	&C + N \xrightarrow[]{b_{i}} 2C,\\
	&rate = b_{i}[C][N]
	\end{align}
\end{subequations}

\begin{equation}
	C_{i} + N_{i} \xrightarrow[]{b_{i}[C_{i}][N_{i}]} 2C_{i}
\end{equation}

\begin{equation}
C + N \xrightarrow[]{b[C][N]} 2C
\end{equation}

\begin{equation}
    -~\delta_{i}
\end{equation}

\begin{subequations}
	\label{eq:9}
	\begin{align}
	&N + C \xrightarrow[]{b_{i}} 2C,\\
	&\frac{dC}{dt} = b_{i}[N][C]
	\end{align}
\end{subequations}

\begin{subequations}
	\label{eq:9}
	\begin{align}
	&r_{i} = b_{i}(N_0 + C_0)\\
	&K_{i} = (N_0 + C_0)
	\end{align}
\end{subequations}

\begin{subequations}
	\label{eq:9}
	\begin{align}
	&r = b(N_0 + C_0)\\
	&K = (N_0 + C_0)
	\end{align}
\end{subequations}


\begin{subequations}
	\label{eq:5}
	\begin{align}
	\frac{dC_{i}}{dt} =&~b_{i}N_{i}C_{i},\\
	\frac{dN_{i}}{dt} =& - b_{i}N_{i}C_{i}\\
	 & - k\sum_{j \epsilon \delta_i}(N_{i} - N_{j})\\
	\end{align}
\end{subequations}

\begin{subequations}
	\label{eq:5}
	\begin{align}
	\frac{dC_{i}}{dt} =&~b_{i}N_{i}C_{i},\\
	\frac{dN_{i}}{dt} =& - b_{i}N_{i}C_{i} - k\sum_{j \epsilon \delta_i}(N_{i} - N_{j})\\
	\end{align}
\end{subequations}

\end{document}
