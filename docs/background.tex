\graphicspath{{images_low_res/}}

\section{Background of Research}
\label{sec:background}

\subsection{A Comparison of Methods for Screening Strain Fitness }


% \begin{figure*}
%   \centering
%   \includegraphics[width=\linewidth]{5658435523_c2e43729f1_b}
%   \captionof{figure}{An example QFA agar.}
% \end{figure*}

\begin{Figure}
  \centering
  \includegraphics[width=\linewidth]{5658435523_c2e43729f1_b}
  \captionof{figure}{An image of 15 cultures on a section of solid agar
    from a QFA procedure. Image taken ??with permission?? from the Colonyzer GitHub repo
    https://github.com/CnrLwlss/Colonyzer \citep{Lawless2010}}
\end{Figure}


\begin{Figure}
  \centering
  \includegraphics[width=\linewidth]{qfa_growth_array}
  \captionof{figure}{Growth curves, captured automatically by
    Colonyzer \citep{Lawless2010}, of 308 cultures in a QFA
    procedure. Reproduced ??with permission?? from \citet{Banks2012}.}
\end{Figure}

\begin{Figure}
  \centering
  \includegraphics[width=\linewidth]{pin_v_spot_growth}
  \captionof{figure}{Fits of the logistic growth model, to cell
    density observations of yeast growing on solid agar, for spotted
    cultures (circles) and pinned cultures (crosses). In A) culture
    density is plotted on a linear scale. In B) culture density is
    plotted on a logarithmic scale. C) shows images of the spotted and
    pinned cultures corresponding to the data in A) \& B). Reproduced
    ??with permission?? from \citet{Lawless2010}.}
\end{Figure}


\subsection{Evidence of Competition and Signalling in QFA}

\subsection{Modelling Approaches}

\subsubsection{Mass Action Kinetics}
\subsubsection{The Logistic Growth Model}
\subsubsection{Fishers multiplicative model of genetic interactions}
\subsubsection{Hierarchical models and Bayesian inference}

\subsection{Inference of Genetic Interactions / Telomere Cap Defects}


%%% Local Variables:
%%% mode: latex
%%% TeX-master: "proposal"
%%% End:
