
\section{Objectives}
\label{sec:objectives}

\begin{itemize}
\item Develop a model of colony/culture growth (on QFA agar) which accounts for
  competition for nutrients and/or signalling between spots.
\item Reanalyse QFA data from (ref) (which investigates genetic
  interactions relating to telomere function in yeast), comparing the
  competition model with the independence model (a special case of the
  competition model), to determine how much competition affects colony
  growth.
\item Compare competition effects in different experimental designs in
  order to design experiments which minimise these effects. (For
  instance, compare spot and agar geometries and initial nutrient
  densities. Also look at barcode design to investigate the effect of
  agar height.)
\item Develop a model of competition in terms of parameters of the
  logistic model of population growth in oder to compare fitness
  measures which account for competition, to measures from previous
  studies which assume independence.
\item Using the competition-adjusted logistic model, reanalyse the
  data from (ref) in order to determine whether modelling
  competition has a significant effect on interaction score and
  whether new interactions can be discovered.
\item Package the model in SBML, conform to minimum information
  standards, and publish in the biomodels database.
\item Document, package, and distribute other models and analysis
  tools (most likely written in python). Conform to minimum
  information standards.
\end{itemize}

%%% Local Variables:
%%% mode: latex
%%% TeX-master: "proposal"
%%% End:
