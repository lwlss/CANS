
\section{Aims}
\label{sec:aims}
Quantitative fitness analysis (QFA) is a method for inferring the
fitness of microbial cultures from their growth curves and can be used
for genome-wide screening of genetic interaction or drug response (see
\citet{Addinall2008,Addinall2011,Lawless2010,Banks2012,Andrew2013}). In QFA, cultures are grown on solid agar plates. When performed
using a high-throughput protocol, growth curves are automatically
captured for mixed arrays of typically 384 (independent genetic
strains)/(cultures). It is possible that growth curves are affected by
competition for nutrients and/or signalling between cultures which is
not accounted for in past analyses of QFA data.

An analysis by \citet{Baryshnikova2010}, of data obtained using
1536-pin synthetic genetic array (SGA), an alternative procedure also
using an array of cultures on solid agar, attempts to normalise for
systematic variation in growth curve observations, which may include
variation from competition and signalling effects, using statistical
techniques aimed at improving the correlation between repeats of
identical mutant strains at different locations. In order to learn
more about the sources of systematic variation in both QFA and SGA
experiments, and to develop analyses and experimental designs for
dealing with them, this project aims to account for competition and
signalling mechanistically. We then aim to determine what effect
accounting for competition and signalling has on measures of
fitness. We will study unpublished QFA and SGA data for the model
organism \textit{Saccharomyces cerevisiae}. We will package all models
and analysis tools so that they may be used in future QFA and SGA
studies or to reanalyse data from past studies.

\subsection{Hypotheses}
This project aims to determine whether
the following hypotheses are correct:
\begin{itemize}
\item %Nutrient and/or signal molecules diffuse through agar at a rate
  %sufficient to induce competition between adjacent cultures.
  Growth of cultures in QFA and SGA experiments is affected by
  competition for nutrients and/or signalling.
\item By accounting mechanistically for competition effects, it will
  be possible to design analyses and experiments which minimise them.
\item Accounting for competition effects will/may improve the power to
  infer genetic interactions.

  ??Are we directly measuring the size of this effect or just saying
  that if fitness measures are affected then this will have some
  effect on genetic interaction measures??
\end{itemize}

%%% Local Variables:
%%% mode: latex
%%% TeX-master: "proposal"
%%% End:
