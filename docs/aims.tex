
\section{Aims}
\label{sec:aims}


Quantitative fitness analysis (QFA) is a method for...  Nutrients are
known to diffuse through agar (ref) and yeast and bacteria are known
to have mechanisms for signalling between cultures/colonies
(ref). There is evidence, from past QFA experiments and related
experiments (ref), that either competition, or signalling, or both are
affecting the growth of colonies in the assays.

An analysis by (ref), of data obtained by an alternative experimental method,
SGA, attempts to account for these effects by normalising measurements
using a statistical approach (be more specific) (non-mechanistically).

% This project aims to investigate whether competition and/or signalling
% is affecting the results of quantitative fitness ananlysis (QFA)
% experiments and how th.
This project aims to investigate the following hypotheses:
%Specifically the aim is to investigate the following hypotheses:
\begin{itemize}
\item Growth of colonies in QFA experiments is not independent and is
  affected by competition and/or signalling.
\item By accounting for competition and signalling mechanistically, it
  will be possible to design experiments which minimise their effect.
\item Reanalysis of QFA data from (ref), using a model accounting for
  competition effects, will better predict genetic interaction
  strength and (possibly) provide evidence of new genetic interactions.
\end{itemize}



%%% Local Variables:
%%% mode: latex
%%% TeX-master: "proposal"
%%% End:
