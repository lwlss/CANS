
\section{Aims}
\label{sec:aims}


Quantitative fitness analysis (QFA) is a method for inferring the
fitness of microbial genetic variants from their growth curves and can be
used for genome-wide screening of genetic interaction or drug response
(see \citet{Addinall2008,Addinall2011,Lawless2010,Banks2012} (drug
ref)). When performed using a high-throughput protocol, growth curves are
automatically captured for mixed arrays of typically 384 independent
genetic strains grown on solid agar. There is evidence, from previous
QFA experiments (ref), that growth of cultures is affected by
competition for nutrients and/or signalling between cultures.
An analysis by \citet{Baryshnikova2010}, of data obtained using synthetic
genetic array (SGA), an alternative procedure also using a solid
agar array, attempts to normalise for systematic variation in growth
curve observations, which may including variation from competition and
signalling effects, using non-mechanistic statistical techniques aimed
at improving the correlation between repeats of identical mutant
strains at different locations. (Could mention advantages
of the Bayesian approach of \citet{Heydari2016} from a statistical and
computational perspective.)

This project aims instead to account for competition and signalling
mechanistically in order to learn more about the sources of systematic
variation in QFA and to develop experimental designs and analyses for
dealing with them. Specifically, the project aims to determine whether
the following hypotheses are correct:
\begin{itemize}
\item Growth of colonies in QFA experiments is affected by competition
  and/or signalling.
\item By accounting mechanistically for competition and signalling effects, it
  will be possible to design experiments which minimise these effects.
\item Reanalysis of QFA data from \citet{Addinall2011}, using a model accounting for
  competition and signalling effects, will better predict genetic interaction
  strength and (may also) provide evidence of new genetic interactions.
\end{itemize}


Motivated by the link between telomere shortening, ageing, and cancer
(refs?), \citet{Addinall2011} study the genetic interactions of
\textit{cdc13} which functions in telomere capping in the model
organism \textit{Saccharomyces cerevisiae}. If the power to predict
genetic interactions is improved in the reanalysis of data from
\citet{Addinall2011}, this may highlight new genes involved in these
processes and will also be relevant for investigations of other types
of genetic interaction and investigations using other types of
microbial organism.


%%% Local Variables:
%%% mode: latex
%%% TeX-master: "proposal"
%%% End:
