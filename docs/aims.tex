
\section{Aims}
\label{sec:aims}
Quantitative fitness analysis (QFA) is a method for inferring the
fitness of microbial cultures from their growth curves and can be used
for genome-wide screening for genetic interactions or drug responses
(see
\citet{Addinall2008,Addinall2011,Lawless2010,Banks2012,Andrew2013}). In
QFA, growth curves are automatically captured for arrays of typically
384 cultures grown on solid agar, with each culture containing an
individual genetic strain. Plates can contain any combination of
different strains or repeats. Fitness is a good phenotype for measuring GI because...

Usually QFA cultures are assumed to grow independently
(e.g. \citet{Addinall2011}). We aim to test the validity of this
assumption in a range of different QFA experiments. We propose a new
model of population growth which includes competition between
cultures. We expect that accounting explicitly for competition
in analysis of QFA data will increase the reproducibility of fitness
estimates and ultimately increase the statistical power of QFA in
screens for genetic interactions and drug responses.

An analysis by \citet{Baryshnikova2010}, of data obtained using
1536-pin synthetic genetic array (SGA), an alternative procedure also
using an array of cultures on solid agar, attempts to normalise for
systematic variation in growth observations, which may include
variation from competition and signalling effects, using statistical
techniques aimed at improving the correlation between repeats of
identical mutant strains at different locations. In order to learn
more about the sources of systematic variation in both QFA and SGA
experiments, and to develop analyses and experimental designs for
dealing with them, this project aims to account for competition and
signalling mechanistically. We then aim to determine what effect
accounting for competition and signalling has on variability and rank
order of fitness estimates. We will study unpublished QFA data for the
model organism \textit{Saccharomyces cerevisiae}. We will package all
models and analysis tools so that they may be used in future studies
or to reanalyse data from past studies.

\subsection{Hypotheses}
This project aims to test the following hypotheses:
\begin{itemize}
\item Growth of cultures in fitness screening experiments using solid
  agar is affected by competition for nutrients and/or signalling.
\item By accounting mechanistically for competition effects, it will
  be possible to design analyses and experiments which minimise these effects.
\item Explicitly accounting for competition effects in data analysis
  will improve the power to infer genetic interactions and drug
  responses.
\end{itemize}

%%% Local Variables:
%%% mode: latex
%%% TeX-master: "proposal"
%%% End:
