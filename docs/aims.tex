
\section{Aims}
\label{sec:aims}


Quantitative fitness analysis (QFA) is a method for inferring the
fitness of microbial genetic variants from their growth curves and can be
used for genome-wide screening of genetic interaction or drug response
(see \citet{Addinall2008,Addinall2011,Lawless2010,Banks2012} (drug
ref)). When performed using a high-throughput protocol, growth curves are
automatically captured for mixed arrays of typically 384 independent
genetic strains grown on solid agar. There is evidence, from previous
QFA experiments (ref), that growth of cultures is affected by
competition for nutrients and/or signalling between cultures.
An analysis by \citet{Baryshnikova2010}, of data obtained using synthetic
genetic array (SGA), an alternative procedure also using a solid
agar array, attempts to normalise for systematic variation in growth
curve observations, which may including variation from competition and
signalling effects, using non-mechanistic statistical techniques aimed
at improving the correlation between repeats of identical mutant
strains at different locations. (Could mention advantages
of the Bayesian approach of Heydari et al.)

This project aims to account for competition and signalling
mechanistically in order to learn more about the sources of systematic
variation in these procedures and to develop experimental designs and
analyses for dealing with them. Specifically, the project aims to
investigate the following hypotheses:
\begin{itemize}
\item Growth of colonies in QFA experiments is affected by competition
  and/or signalling.
\item By accounting mechanistically for competition and signalling effects, it
  will be possible to design experiments which minimise these effects.
\item Reanalysis of QFA data from \citet{Addinall2011}, using a model accounting for
  competition and signalling effects, will better predict genetic interaction
  strength and (may also) provide evidence of new genetic interactions.
\end{itemize}


\citet{Addinall2011} aim to discover genes interacting with a
temperature-sensitive mutant of \textit{cdc13} in
\textit{Saccharomyces cerevisiae} because of \textit{cdc13}'s function in telomere
capping and the link between telomere shortening, ageing, and cancer.


If the power to predict genetic interactions is improved in the
reanalysis of data from \citet{Addinall2011}, this may provide clues
t will be
relevant for investigations of other types of genetic interaction and
investigations using other types of microbial organism.


%%% Local Variables:
%%% mode: latex
%%% TeX-master: "proposal"
%%% End:
