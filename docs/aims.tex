
\section{Aims}
\label{sec:aims}


Quantitative fitness analysis (QFA) is a method for inferring the
fitness of microbial genetic variants from their growth curves (see
refs). In this approach, genetic variants are grown on solid agar in
an array up to (number) colonies. Nutrients are known to diffuse
through agar (ref) and microbial organisms, including
\textit{Saccharomyces cerevisiae}, are known to have mechanisms for
signalling between cultures/colonies (ref). There is evidence, from
past QFA experiments (ref) and SGA experiments
\citep{Baryshnikova2010}, that either competition, or signalling, or
both is affecting the growth of colonies in the assays. An analysis by
\citet{Baryshnikova2010}, of data from synthetic genetic arrays
(SGAs), a related experimental method, attempts to normalise for
systematic variation, including from competition effects, using
statistical techniques aimed at improving the correlation between
repeats of identical mutant strains at different locations. This is a
non-mechanistic solution. (Could mention advantages of the Bayesian
approach of Heydari et al.)

% This project aims to investigate whether competition and/or signalling
% is affecting the results of quantitative fitness ananlysis (QFA)
% experiments and how th.
This project aims to account for competition and signalling
mechanistically in order to investigate the following hypotheses:
This project aims to investigate the following hypotheses:
%Specifically the aim is to investigate the following hypotheses:
\begin{itemize}
\item Growth of colonies in QFA experiments is not independent and is
  affected by competition and/or signalling.
\item By accounting mechanistically for competition and signalling effects, it
  will be possible to design experiments which minimise these effects.
\item Reanalysis of QFA data from (ref), using a model accounting for
  competition effects, will better predict genetic interaction
  strength and (possibly) provide evidence of new genetic interactions.
\end{itemize}


Telomeres (the whatever gene) are believed involved cancer and aging
(ref). Knowledge of interactions could improve our understanding of...

If, in the data from (ref), modelling competition improves the
prediction of genetic interactions, this is also likely to be relevant
to assays of other mibrobial organisms and investigations of other
types or genetic interaction.


%%% Local Variables:
%%% mode: latex
%%% TeX-master: "proposal"
%%% End:
